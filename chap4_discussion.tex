\chapter{Discussion}
  \label{chap:Discussion}
  \section{A Correlation is not Enough}
     As noted by Jones at al. (2013) at the \textit{Life Cycle of Cosmic Dust} conference in Taipei, material in the galaxy tends to correlate with other material in the galaxy. In other words, the greatest challenge of this type of study is not in finding a correlation between PAHs or any other dust component and  AME, indeed such a correlation is expected and easy to find. The challenge is in finding second-order variations that relate to the AME and spinning dust. For example, even if we had a truly perfect survey of pure PAH emission in the galaxy, finding that it correlates with AME would not be particularly useful. We expect that dense regions of the ISM will contain all major components of dust / gas in some proportions. There may be regions where PAHs, for example, are perhaps being destroyed due to environmental conditions, but in general the major components of dust and gas will trace one another. 
     What we need to find, in order to make any clear conclusion, is how the different components vary with respect to one another. This challenge has certainly presented itself in this thesis project. While the notion of a possible PAH-AME connection was motivating, it is not so easy to establish. There are a few fundamental questions involved:
  
  1.) What influences the UIR/PAH bands? Do ``pure" PAHs exist in significant abundances? 
  
     A mixture of PAHs and other PAH-class aromatic molecules could mean a wide mixture of AME carriers, perhaps being rotationally excited to different extents under different conditions. This could mean that even if the PAH band carriers are also the carriers of the AME, a clear correlation with the AME may remain elusive. 
  
  2.) Are the UIR / PAH band carriers also the carriers of the AME? Are they the only carriers?

    Even if the AME is spinning dust, there is the possibility that the PAH band carriers are not the AME carriers. If they are, they may not be the only carriers. For example, other very small grains not strongly covered by a PAH dominated survey like AKARI/IRC 9~$\mu$m could be contributing spinning dust emission.
  
  3.) What is the major contributor to the AME observed in these regions?
  
  Even if we are able to understand completely the nature of the PAH bands emission, it may still be the case that we do not fully understand the mechanisms which can produce AME. For example, the magnetic dust contribution is still uncertain. Free-free emission is also sometimes difficult to quantify. For the diffuse sky, it has been shown that there is not a clearly explained model that can explain the AME \citep{hfi14viii}.
  
  4.) What are the environmental factors that influence AME?
  
     We must remember that simply the presence of possible carriers of AME may not influence the observed AME. Whether spinning dust or magnetic dust, if environmental conditions like ISRF, column density, or metallicity are more important for producing an environment that promotes AME, then it is not sufficient only to identify the carriers. In this way too, me may not see any especially strong relationship among even a PAH-survey. 
     With these questions in mind, we have performed an examination of the data across an expansive wavelength range. The maps and instruments involved are very advanced and in some cases quite specialized in their objective, like the PAH band tracing AKARI/IRC 9~$\mu$m survey \citep{irc07,ishihara10}, or the peak thermal emission tracing FIS surveys \citep{doi12}, or the Planck/HFI data tracing primarily cold dust. We discuss the results of this comparison below. This is the first time AKARI, IRAS, and HFI data have been combined for AME studies.
  
  \section{$A_{sp}$ vs. $\tau_{250}$}
     The moderate trend found in Chapter 3 between $\tau_{250}$ and $A_{sp}$ suggests a dust-AME relationship (R = 0.87) for AME regions. This result is as expected, assuming that AME arises from dust. As was described by \cite{draine98a} via Equation \ref{eq:Pspin} that more spinning grains would directly relate to a higher amplitude of spinning dust emission. Our result follows the result of both PCXV and \cite{ysard10a}. We find that the result is similar non-AME regions (R=0.77), in agreement with PCXV.
     In the case of non-AME regions,  we must use some caution. The authors of PCXV emphasize that $A_{sp}$ is only valid as a measure of column density of spinning dust if their model is correct for a given ROI. If the regions from which $A_{sp}$ is determined do not have an SED that agrees with the spinning dust model, the correlation test result has less meaning. In such a case, $A_{sp}$ is more likely to trace free-free emission or thermal dust emission which was not completely removed from the microwave data (``residual residual" emission). However, even in this case we would expect that $A_{sp}$ should increase with $\tau_{250}$. We note that the correlation of $\tau_{250}$ with $A_{sp}$ is stronger for significant AME regions. However, this may not be a significant difference as the error bars of $A_{sp}$ for non-AME regions are larger by definition. Moreover, the sample size of non-AME regions is larger than the sample size of AME regions. 
     Perhaps a re-determination of $A_{sp}$ and $\sigma$AME for each ROI, with adjustments made to the spinning dust model on a region-by-region basis, could more powerfully test for a $\tau_{250}$ vs. AME relationship. Another option is to repeat the study with full-SED modelling. As of the time of this writing, template dust models for the model used by \cite{galliano11} should soon be available. We hope to include a full-SED modelling based on these templates, as well as modelling based on the DustEM code by \cite{dustem11}. In this way we can obtain the relative abundances of each of the major dust components (aromatics, carbonaceous grains, silicate grains) simultaneously. This would offer a more powerful comparison of dust properties vs. the AME. 

\section{Comparison of AME with IRC Data}
     The full dust SED modelling seems to be even more necessary, given the result of the MIR vs. $A_{sp}$ plots. A spinning dust model suggests that the correlations should become clearer after dividing $I_{MIR}(\lambda)$ by $G0$. This is because the spinning dust intensity is not expected to depend on $G0$. Also, dividing $I_{MIR}(\lambda)$ by $G0$ should more closely trace the abundance of PAHs and VSGs. Thus the trend should be more clear. However we find that the correlation between MIR and $A_sp$ becomes much worse after scaling by $G0$. This suggests perhaps that the spinning PAH hypothesis is not correct, or at least that the role of $G0$ in producing spinning dust may be more important than initially thought.
     \cite{tibbs12b} find that the AME in the HII region RCW175 does not show a correlation between AME and small dust emission. PCXV find that RCW175 does not have a high AME significance in the Planck data, thus we cannot compare these results to \cite{tibbs12b} directly. However we have shown that the general trend for the 98 ROIs is that a weaker correlation is seen with the short-wavelength bands (both the 9~$\mu$m survey which traces PAHs heavily and those which trace primarily other VSG emission, like IRAS 12~$\mu$m and AKARI 18~$\mu$m vs. bands dominated by emission from the FIR modified blackbody. 
 
\subsection{$\tau_{250}$ vs. $A_{sp}$ Compared to AKARI 9~$\mu$m vs. AME}
     The lack of a clear PAH-band vs. $A_{sp}$ trend is especially interesting. This could indicate that non-PAH small grains are more dominant in carrying the AME, or that the simple averaged 9~$\mu$m intensity is not sufficient to access a relation with AME. A possible systematic explanation for the weaker correlation in the IRC data is the lack of a sophisticated zody-subtraction. Background subtraction was completed for the IRC data using a robust-mean subtraction (see Chapter \ref{chap:Data}). The zody model is still being perfected by Ishihara et al. (in prep.). Some of the calibration parameters of the IRC maps are also being redetermined. In any case we will re-run the analysis when the final IRC all-sky maps are released.     
 %     PCXV carried out their IR analysis IRAS bands only. Our result for the IRAS bands is similar to theirs. Variations may be due to the $G0$ and $A_{sp}$ values having been determined at much larger angular resolutions, with all of their data being smoothed to 1$\degree$.
  
\subsection{Magnetic dust emission: A remaining question}
     The magnetic dipole emission interpretation for AME cannot be ruled out, based on our analysis. Magnetic dust may be a substitute explanation for the AME, or possibly an overlapping explanation. Magnetic dipole emission from dust does not require PAHs. If the AME is connected more to $\tau_{250}$ than to $\tau_{MIR}$, that temperature fluctuations of magnetic grains is an interpretation that mus be explored further. We cannot offer a strong conclusion about magnetic dust here, but we leave it as an open question for future research.
