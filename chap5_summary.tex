\chapter{Summary
  \label{chap:Summary}}
     This thesis has presented one of the earliest efforts to conduct a multi-wavelength investigation using all 7 AKARI all-sky surveys along with IRAS and Planck data. The wavelength range spans 9~$\mu$m to 545~$\mu$m. The purpose was to gather information from all major dust components, from aromatic organic molecules (Section \ref{PAHhypothesis}) to larger carbonaceous and silicate grains. Chapter \ref{chap:Introduction} gives a general overview of the ISM and interstellar dust, and the basic physics of the AME and galactic foreground.
     A total of 98 regions from PCXV were examined for the first time at AKARI wavelengths. Most importantly, the AKARI/IRC 9~$\mu$m all-sky data was inspected. This survey contains one of the most complete data-sets for looking at the PAH/UIR bands over a very large spatial range. The PAH band photometry was considered critical because one of the primary suspects for an AME carrier is the PAH family of molecules (Section \ref{spinningdust}).
     Chapter \ref{chap:Data} reviews the data sources used for this study. The extraction of sources and their processing are also described, including the PSF smoothing, pixel regridding, SED extraction, modified blackbody fitting, and color-correction. A major component of the data processing is the smoothing of the data to the largest PSF, Planck/HFI 545~GHz at 4.7$'$. A major limitation of the processing is that the finer resolution of AKARI/IRC and FIS is not able to be utilized due to the PSF smoothing. We can take advantage only of the overall spatial coverage, and additional sampling positions along the dust SED in the MIR and FIR (especially in the PAH domain, and around the peak of the thermal emission curve). 
%A case study of one region, $\rho$~Ophiuchi is presented to demonstrate the resolving capability of AKARI (and its uniqueness as an all-sky data set at such fine resolution).
     Chapter 3 presents several attempts to illuminate possible patterns between the IR photometry and the AME/spinning dust parameters derived by PCXV which examined lower-frequency data not used in this thesis, i.e. Planck/LFI \citep{lfi14ii}. The methodology for deriving estimates of dust properties from the SEDs is presented, such as the optical depth at 250~$\mu$m (Equation 3.1.1). Plots of each photometric band's average intensity against the AME amplitude derived by PCXV are presented. The same plots, though with the IR intensities divided by the ISRF scaling factor $G0$ are also included.
     In Chapter \ref{chap:Discussion}, a lack of support for a spinning dust model in the current analysis is described. Ways in which the data might allow for a magnetic dust contribution is described, as well as a possibility that the results are due to varying methods between the current analysis and that of preceding works, PCXV and \cite{ysard10a}. The contradicting results of this study and the same analysis performed by the previous works' authors is of the most importance. Prior to this analysis it was generally accepted that shorter wavelength data correlates more tightly with AME than longer wavelength dust SED data. We do not find the same pattern, especially in the case of the AKARI/IRC 9~$\mu$m. We also fail to find that the correlations become stronger after scaling by $G0$.
     In the same way, the plots which show FIR-emitting dust optical depth $\tau_{250}$ vs. the AME indicate a general trend between dust column density and AME amplitude. A slightly stronger trend is seen with the significant AME regions than with non-significant AME regions. This result agrees with that of previous works. The implications were discussed, as well as the disagreement with the IRC data. The question of contribution of magnetic dipole emission from dust to the AME cannot be answered here, but we do argue that it is necessary to fully explore the magnetic dust hypothesis.